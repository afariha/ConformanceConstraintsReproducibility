%!TEX root = paper.tex

\looseness-1 
%Data is central to modern systems in a wide range of domains.
%, including healthcare, transportation, and finance. 
The core of modern data-driven systems typically comprises of models learned
from large datasets, and they are usually optimized to target particular data
and workloads. While these data-driven systems have seen wide adoption and
success, their reliability and proper functioning hinge on the data's continued
conformance to the systems' initial settings and assumptions. When the serving
data (on which the system operates) deviates from the profile of the initial
data (on which the system was trained), system performance degrades and system
behavior becomes unreliable. Therefore, a mechanism to assess the
trustworthiness of a system's inferences is paramount, especially for systems
that perform safety-critical or high-impact operations.

A machine-learned (ML) model typically works best if the serving dataset
follows the profile of the dataset the model was trained on; when it doesn't,
the model's inference can be unreliable. One can profile a dataset in many
ways, such as by modeling the data distribution of the
dataset~\cite{achlioptas2017learning}, or by finding the (implicit)
\emph{constraints} that the dataset satisfies~\cite{pena2019discovery}.
Distribution-oriented approaches learn data likelihood (e.g., joint or
conditional distribution) from the training data, and can be used to check if
the serving data is unlikely. An unlikely tuple does not necessarily imply that
the model would fail for it. The problem with the distribution-oriented
approaches is that they tend to overfit, and, thus, are overly conservative
towards unseen tuples, leading them to report many such false positives.

\looseness-1 We argue that certain constraints offer a more effective and
robust mechanism to quantify trust of a model's inference on a serving tuple.
The reason is that learning systems implicitly exploit such constraints during
model training, and build models that assume that the constraints will continue
to hold for serving data. For example, when there exist high correlations among
attributes in the training data, learning systems will likely reduce the
weights assigned to redundant attributes that can be deduced from others, or
eliminate them altogether through dimensionality reduction. If the serving data
preserves the same correlations, such operations are inconsequential;
otherwise, we may observe model failure.

\looseness-1 In this paper, we characterize datasets with a new data-profiling
primitive, \emph{\dis}, and we present a mechanism to identify \emph{strong}
\dis, whose violation indicates unreliable inference. \Dis specify constraints
over \emph{arithmetic relationships} involving multiple numerical attributes of
a dataset. We argue that a tuple's conformance to the \dis is more critical for
accurate inference than its conformance to the training data distribution. This
is because any violation of \dis is likely to result in a catastrophic failure
of a learned model that is built upon the assumption that the \dis will always
hold. Thus, we can use a tuple's deviation from the \dis as a proxy for the
trust on a learned model's inference for that tuple. We proceed to describe a
real-world example of \dis, drawn from our case-study evaluation on
\emph{trusted machine learning} (TML).

\setlength{\tabcolsep}{.5em}
\renewcommand{\arraystretch}{.9}
\begin{figure}[t]
\centering
    \resizebox{1\columnwidth}{!}
	{\small
	
    \begin{tabular}{lcccc}
    \toprule
     							& \textbf{Departure} 		& \textbf{Departure Time} 		& \textbf{Arrival Time} 		& \textbf{Duration (min)} 	\\
								& \textbf{Date} 	 		& \textbf{[DT]} 				&  \textbf{[AT]} 				& \textbf{[DUR]} 		  	\\
    \midrule
		$t_1$ 					& May 2 					& \texttt{14:30} 				& \texttt{18:20}  				& 230 					  	\\
		$t_2$ 					& July 22 					& \texttt{09:05}				& \texttt{12:15}  				& 195 					  	\\
        \revisetwo{$t_3$} 		& 	\revisetwo{June 6} 		& \revisetwo{\texttt{10:20}}	  & \revisetwo{\texttt{12:20}}  & \revisetwo{115} 			\\ 
		$t_4$ 					& May 19 					& \texttt{11:10} 				& \texttt{13:05}  			   	& 117 						\\
		\rowcolor{vlightgray}
		$t_5$ 					& April 7 					& \texttt{22:30}  			  & \texttt{06:10}  			  	& 458						\\

    \bottomrule
    \end{tabular}
    }
	\vspace{-3mm}
    \caption{\small
	%
	 Sample of the airlines dataset (details are in
	 Section~\ref{exp-invariants-for-ML}), showing departure, arrival, and
	 duration only. The dataset does not report arrival date, but an arrival time
	 earlier than departure time (e.g., last row), indicates an overnight flight.
	 \reviseone{All times are in 24 hour format} and in the same time zone. There is some
	 noise in the values.
	%
	} 
	\vspace{-2mm}
    \label{fig:flights}
\end{figure}

\begin{example}\label{ex:tml}
	\looseness-1

We used a dataset with flight information that includes data on departure and
arrival times, flight duration, etc.\ (Fig.~\ref{fig:flights}) to train a
linear regression model to predict flight delays. \revisetwo{The model was
trained on a subset of the data that happened to include only daytime flights
(such as the first four tuples)}. In an empirical evaluation of the regression
accuracy, we found that the mean absolute error of the regression output more
than quadruples for overnight flights (such as the last tuple $t_5$), compared
to daytime flights. The reason is that tuples representing overnight flights
deviate from the profile of the training data \revisetwo{that only contained
daytime flights}. Specifically, daytime flights satisfy the \di that ``arrival
time is later than departure time and their difference is very close to the
flight duration'', which does not hold for overnight flights. Note that this
\invariant is just based on the covariates (predictors) and does not involve
the target attribute $\mathtt{delay}$. Critically, although this \di is unaware
of the regression task, it was still a good proxy of the regressor's
performance. \revisetwo{In contrast, approaches that model data likelihood may
report long daytime flights as unlikely, since all flights in the training data
($t_1$--$t_4$) were also short flights, resulting in false alarms, as the model
works very well for most daytime flights, regardless of the duration (i.e., for
both short and long daytime flights).}
	%
\end{example}

\revisetwo{Example~\ref{ex:tml} demonstrates that when training data has
\emph{coincidental} relationships (e.g., the one between $\mathtt{AT}$,
$\mathtt{DT}$, and $\mathtt{DUR}$ for daytime flights), then ML models may
\emph{implicitly} assume them as \emph{invariants}. \Dis can capture such data
invariants and flag non-conforming tuples (overnight flights) during
serving.}\label{extake}

\smallskip

\paragraph{\Dis.} \Dis complement the existing data profiling literature, as
the existing constraint models, such as functional dependencies and denial
constraints, cannot model arithmetic relationships. For example, the \di of
Example~\ref{ex:tml} is: $-\epsilon_1 \le \mathtt{AT} - \mathtt{DT} -
\mathtt{DUR} \le \epsilon_2$, where $\epsilon_1$ and $\epsilon_2$ are small
values. \Dis can capture complex linear dependencies across attributes within a
\emph{noisy} dataset. For example, if the flight departure and arrival data
reported the hours and the minutes across separate attributes, the \invariant
would be on a different arithmetic expression: $(60\cdot \mathtt{arrHour} +
\mathtt{arrMin}) - (60\cdot \mathtt{depHour} + \mathtt{depMin}) -
\mathtt{duration}$.

\looseness-1 The core component of a \di is the arithmetic expression, called
\emph{projection}, which is obtained by a linear combination of the numerical
attributes. There is an unbounded number of projections that we can use to form
arbitrary \dis. For example, for the projection $\mathtt{AT}$, we can find a
broad range $[\epsilon_3, \epsilon_4]$, such that all training tuples in
Example~\ref{ex:tml} satisfy the \di $\epsilon_3 \le \mathtt{AT} \le
\epsilon_4$. However, this \invariant is too inclusive and a learned model is
unlikely to exploit such a weak constraint. In contrast, the projection
$\mathtt{AT} - \mathtt{DT} - \mathtt{DUR}\;$ leads to a stronger \di with a
narrow range as its bounds, which is selectively permissible, and, thus, more
effective.

\smallskip

\paragraph{Challenges and solution sketch.} The principal challenge is to
discover an \emph{effective} set of conformance constraints that are likely to
affect a model's inference implicitly. We first characterize ``good''
projections (that construct effective constraints) and then propose a method to
discover them. We establish through theoretical analysis two important results:
(1)~A projection is good over a dataset if it is almost constant (i.e., has low
variance) over all tuples in that dataset. (2)~A set of projections,
collectively, is good if the projections have small pair-wise correlations. We
show that low variance components of a principal component analysis (PCA) on a
dataset yield such a set of \views. Note that this is different from---and, in
fact, completely opposite of---the traditional approaches
(e.g.,~\cite{DBLP:conf/kdd/QahtanAWZ15}) that perform multidimensional analysis
based on the high-variance principal components, after reducing dimensionality
using PCA.

\smallskip

\paragraph{Scope.} \looseness-1 Fig.~\ref{relatedWorkMatrix} summarizes prior
work on related problems, but our scope differs significantly. Specifically, we
can detect if a serving tuple is non-conforming with respect to the training
dataset \emph{only based on its predictor attributes}, and require no knowledge
of the ground truth. This setting is essential in many practical applications
when we observe \emph{extreme verification latency}~\cite{souzaSDM:2015}, where
ground truths for serving tuples are not immediately available. For example,
consider a self-driving car that is using a trained controller to generate
actions based on readings of velocity, relative positions of obstacles, and
their velocities. In this case, we need to determine, only based on the sensor
readings (predictors), when the driver should be alerted to take over vehicle
control.
%, as we cannot use ground-truths to generate an alert.
%
Furthermore, we \emph{do not assume access to the model}, i.e., model's
predictions on a given tuple. This setting is necessary for (1)~safety-critical
applications, where the goal is to quickly alert the user, without waiting for
the availability of the prediction, (2)~auditing and privacy-preserving
applications where the prediction cannot be shared, and (3)~when we are unaware
of the detailed functionality of the system due to privacy concerns or lack of
jurisdiction.
% , but only have some
% meta-information such as the system trains some linear model over the training
% data.
%
We focus on identifying \emph{tuple-level} non-conformance as opposed to
dataset-level non-conformance that usually requires observing entire data's
distribution. However, our tuple-level approach trivially extends (by
aggregation) to the entire dataset.

%!TEX root = paper.tex

\newcolumntype{?}{!{\vrule width 1pt}}
\newcommand{\sln}{2}
\newcommand{\eln}{18} 
\newcommand{\cm}{\checkmark}
\newcommand{\gm}{!}
\let\st\relax
\newcommand{\st}{$\star$}
\newcommand{\na}{$\bot$}
\newcommand{\nr}{$\varoslash$}
\newcommand*{\bc}{% bold checkmark
  \textpdfrender{
    TextRenderingMode=FillStroke,
    LineWidth=.5pt,
  }{\checkmark}
}

\begin{figure}[t] 
	\setlength\tabcolsep{1pt} 
	\centering	
	%\renewcommand{\arraystretch}{1.0}
	\resizebox{1\columnwidth}{!}{ \small 
	\begin{tabular}
		{?c|c|p{50mm}?c|c|c|c|c|c?p{3mm}|c?c|c|c?c|c|c|c?c|c?} 
		
		\hline
		
		\hline
		
		\multicolumn{3}{?c?}{\cellcolor{vlightgray}{\multirow[c]{1}{*}{\cellcolor{vlightgray}{\rotatebox[origin=lb]{0}{{\textbf{Legend}}}}}}} & 
		
		% High-level Aspects
		\multicolumn{6}{c?}{{\multirow[c]{1}{*}{\rotatebox[origin=lb]{0}{{\scriptsize constraints}}} }} &
		\multicolumn{2}{c?}{{\multirow[c]{1}{*}{\rotatebox[origin=lb]{0}{{\scriptsize violation}}}}} &
		\multicolumn{3}{c?}{{\multirow[c]{1}{*}{\rotatebox[origin=lb]{0}{{\scriptsize setting}}}}} &
		\multicolumn{4}{c?}{{\multirow[c]{1}{*}{\rotatebox[origin=lb]{0}{{\scriptsize technique}}}}} &
		\multicolumn{2}{c?}{{\multirow[c]{1}{*}{\rotatebox[origin=lb]{0}{{\scriptsize TML}}}}}\\
		\hline
		
		\hline

		% Legends
		\multicolumn{2}{?r}{\cellcolor{vlightgray}{\small HP:	}} & 			\multicolumn{1}{l?}{\cellcolor{vlightgray}\small Hyper Parameter } &&&&&&&&&&&&&&&&&\\
		\multicolumn{2}{?r}{\cellcolor{vlightgray}{\small FD:	}} & 			\multicolumn{1}{l?}{\cellcolor{vlightgray}\small Functional Dependency} &&&&&&&&&&&&&&&&&\\
		\multicolumn{2}{?r}{\cellcolor{vlightgray}{\small DC:	}} & 			\multicolumn{1}{l?}{\cellcolor{vlightgray}\small Denial Constraint}		&&&&&&&&&&&&&&&&&\\
		\multicolumn{2}{?r}{\cellcolor{vlightgray}{\small \nr:	}} & 			\multicolumn{1}{l?}{\cellcolor{vlightgray}\small Does not require } &&&&&&&&&&&&&&&&&\\
		\multicolumn{2}{?r}{\cellcolor{vlightgray}{\small \na:	}} & 			\multicolumn{1}{l?}{\cellcolor{vlightgray}\small Not applicable } &&&&&&&&&&&&&&&&&\\
		\multicolumn{2}{?r}{\cellcolor{vlightgray}{\small \st:	}} & 			\multicolumn{1}{l?}{\cellcolor{vlightgray}\small Supports via extension} &&&&&&&&&&&&&&&&&\\
		\multicolumn{2}{?r}{\cellcolor{vlightgray}{\small !:	}} & 			\multicolumn{1}{l?}{\cellcolor{vlightgray}\small Partially}& 		 
		
		% Constraints aspects	 
		\multirow[t]{4}{*}{\begin{sideways}{{\small parametric}}\end{sideways}} &
		\multirow[t]{4}{*}{\begin{sideways}{{\small arithmetic}}\end{sideways}} &
		\multirow[t]{4}{*}{\begin{sideways}{{\small approximate}}\end{sideways}} &	
		\multirow[t]{4}{*}{\begin{sideways}{{\small conditional}}\end{sideways}}&
		\multirow[t]{4}{*}{\begin{sideways}{{\small notion of weight}}\end{sideways}} &
		\multirow[t]{4}{*}{\begin{sideways}{{\small interpretable}}\end{sideways}} &

		% Violation aspects
		\multirow[t]{4}{*}{\begin{sideways}{{\small continuous}}\end{sideways}} &
		\multirow[t]{4}{*}{\begin{sideways}{{\small tuple-wise}}\end{sideways}} &

		% Setting aspects
		\multirow[t]{4}{*}{\begin{sideways}{{\small noisy data}}\end{sideways}} &
		\multirow[t]{4}{*}{\begin{sideways}{{\small numerical attr.}}\end{sideways}} &
		\multirow[t]{4}{*}{\begin{sideways}{{\small categorial attr.}}\end{sideways}} &
		
		% Technique aspects
		\multirow[t]{4}{*}{\begin{sideways}{{\small $\!\varoslash\!$ thresholds}}\end{sideways}} &
		\multirow[t]{4}{*}{\begin{sideways}{{\small $\!\varoslash\!$ distance metric}}\end{sideways}} &
		\multirow[t]{4}{*}{\begin{sideways}{{\small $\!\varoslash\!$ HP tuning}}\end{sideways}}&
		\multirow[t]{4}{*}{\begin{sideways}{{\small scalable}}\end{sideways}} &

		% TML aspects
		\multirow[t]{4}{*}{\begin{sideways}{{\small task agnostic}}\end{sideways}} &
		\multirow[t]{4}{*}{\begin{sideways}{{\small $\!\varoslash\!$ access to model}}\end{sideways}}\\
		
		\hline 
		
		\hline
		
		% Data Profiling techniques
		\multirow{13}{*}{\begin{sideways}{Data Profiling}\end{sideways}} & 
		 
						   \multicolumn{2}{l?}{\textbf{\DIs} } 																&\bc&\bc&\bc&\bc&\bc&\st&	\multicolumn{1}{c|}{\bc} &\bc&	\bc&\bc	&\bc&		\bc&\bc &\bc&\bc&		\bc&\bc\\
		\cline{\sln-20}& \multicolumn{2}{l?}{FD~\cite{papenbrock2015functional}} 											&	&	&	&	&	&\cm&		&	&	   &	&\cm&	\cm&\cm &\cm&   &		\multicolumn{2}{c?}{\multirow{12}{*}{\begin{sideways}{not addressed in prior work}\end{sideways}}}\\
		\cline{\sln-\eln}& \multicolumn{2}{l?}{Approximate FD~\cite{kruse2018efficient}} 									&	&	&\cm&	&	&\cm&		&	&	\cm&	&\cm&	   &\cm &\cm&	&	  \multicolumn{2}{c?}{}\\
		\cline{\sln-\eln}& \multicolumn{2}{l?}{Metric FD~\cite{koudas2009metric}} 											&	&	&\cm&	&	&\cm&		&	&	\cm&\cm	&\cm&	   &	&\na&\na&	  \multicolumn{2}{c?}{}\\
		\cline{\sln-\eln}& \multicolumn{2}{l?}{Conditional FD~\cite{DBLP:conf/icde/FanGLX09}} 								& ! &	&	&\cm&	&\cm&		&\cm&	\cm&	&\cm&	   &\cm &\cm&   &	  \multicolumn{2}{c?}{}\\
		\cline{\sln-\eln}& \multicolumn{2}{l?}{Pattern FD~\cite{qahtan2020pattern}} 										& ! &	&	&	&	&\cm&		&\cm&	\cm&	&\cm&	   &\cm &\cm&   &	  \multicolumn{2}{c?}{}\\		
		\cline{\sln-\eln}& \multicolumn{2}{l?}{Soft FD~\cite{DBLP:conf/sigmod/IlyasMHBA04}} 								&	&	&\cm&	&\cm&\cm&	\cm &	&	\cm&\cm	&\cm&	   &\cm &\cm&\cm&	  \multicolumn{2}{c?}{}\\
		\cline{\sln-\eln}& \multicolumn{2}{l?}{Relaxed FD~\cite{caruccio2016discovery}} 									&	&	&\cm&	&	&\cm&		&	&	\cm&\cm	&\cm&	   &	&\cm&	&	  \multicolumn{2}{c?}{}\\
		\cline{\sln-\eln}& \multicolumn{2}{l?}{FDX~\cite{DBLP:conf/sigmod/ZhangGR20}} 										&	&	&	&	&	&\cm&		&	&	\cm&	&\cm&	   &	&	&\cm&	  \multicolumn{2}{c?}{}\\
		\cline{\sln-\eln}& \multicolumn{2}{l?}{Differential Dependency~\cite{song2011differential}} 						&	&	&	&	&	&\cm&		&	&	   &\cm	&\cm&	   &	&\cm&	&	  \multicolumn{2}{c?}{}\\
		\cline{\sln-\eln}& \multicolumn{2}{l?}{DC~\cite{DBLP:journals/pvldb/ChuIP13, DBLP:journals/pvldb/BleifussKN17}} 	& ! &	&\cm&\cm&\cm&\cm&		&\cm&	\cm&\cm	&\cm&	   &\cm &\cm&	&	  \multicolumn{2}{c?}{}\\
		\cline{\sln-\eln}& \multicolumn{2}{l?}{Approximate DC~\cite{pena2019discovery, DBLP:journals/corr/abs-2005-08540}} 	& ! &	&\cm&\cm&\cm&\cm&		&\cm&	\cm&\cm	&\cm&	   &\cm &\cm&	&	  \multicolumn{2}{c?}{}\\
		\cline{\sln-\eln}& \multicolumn{2}{l?}{Statistical Constraint~\cite{DBLP:conf/sigmod/YanSZWC20}} 					&	&	&\cm&	&	&\cm&	\cm	&	&	\cm&\cm &\cm&	   &\cm &\cm&\cm&	  \multicolumn{2}{c?}{}\\
			
		\hline
						
		\hline
		
		% Learning Techniques	
		\multirow{7}{*}{\begin{sideways}{Learning}\end{sideways}} & 
						   \multicolumn{2}{l?}{Ordinary Least Square} 														&\cm&\cm&\cm&	&	&\st		&\cm&\cm		&\cm&\cm&\st		&\cm&\cm&\cm&\cm		&	&	\\
		\cline{\sln-20}& \multicolumn{2}{l?}{Total Least Square} 															&\cm&\cm&\cm&	&	&\st		&\cm&\cm		&\cm&\cm&\st		&\cm&\cm&\cm&\cm		&\cm&\cm\\
		\cline{\sln-20}& \multicolumn{2}{l?}{Auto-encoder~\cite{DBLP:journals/corr/abs-1812-02765}} 						&\multicolumn{6}{c?}{\na}		&\cm&\cm		&\cm&\cm&\cm		&	&	&	&\cm		&\cm&\cm\\
		\cline{\sln-20}& \multicolumn{2}{l?}{Schelter et al.~\cite{DBLP:conf/sigmod/SchelterRB20}\textsuperscript{+}} 	&\multicolumn{6}{c?}{\na}		&\cm&			&	&\cm&\cm		&	&\cm&\cm&\cm		&	&	\\
		\cline{\sln-20}& \multicolumn{2}{l?}{Jiang et al.~\cite{DBLP:conf/nips/JiangKGG18}} 								&\multicolumn{6}{c?}{\na}		&\cm&\cm		&\cm&\cm&\cm		&\cm&	&	&\cm		&	&	\\
		\cline{\sln-20}& \multicolumn{2}{l?}{Hendrycks et al.~\cite{DBLP:journals/corr/HendrycksG16c}} 					&\multicolumn{6}{c?}{\na}		&	&\cm		&\cm&\cm&\cm		&\cm&\cm&\cm&\cm		&	&	\\
		\cline{\sln-20}& \multicolumn{2}{l?}{Model's Prediction Probability} 												&\multicolumn{6}{c?}{\na}		&\cm&\cm		&\multicolumn{7}{c?}{varies}				&   &	\\
		\hline
		
		\hline
		\multicolumn{20}{r}{\textsuperscript{+} {\scriptsize Requires additional information}}\\

	\end{tabular}
	} 	
	\vspace{-4mm} 	
	\caption{\Dis complement existing data profiling primitives and provide an 
	efficient mechanism to quantify trust in prediction, with
	minimal assumption on the setting. 
    } 
	\label{relatedWorkMatrix} 
	\vspace{-4mm} 
	
\end{figure}


\smallskip

\paragraph{Contrast with prior art.} We now discuss where \dis fit with respect
to the existing literature (Fig.~\ref{relatedWorkMatrix}).
% on data profiling and literature on modeling trust in data-driven inferences 

\looseness-1 \subsubsection*{Data profiling techniques} \Dis fall under the
umbrella of data profiling, which refers to the task of extracting~tech\-nical
metadata about a given dataset~\cite{DBLP:journals/vldb/AbedjanGN15}. A key
task in data profiling is to learn relationships among attributes. Functional
dependencies (FD)~\cite{papenbrock2015functional} and their variants only
capture if a relationship exists between two sets of attributes, but do not
provide a closed-form (parametric) expression of the relationship. Using the FD
``$\{\mathtt{AT}, \mathtt{DT}\} \rightarrow$ $\{\mathtt{DUR}\}$'' to model the
\invariant of Example~\ref{ex:tml} suffers from several limitations. First,
since the data is noisy, no exact FD can be learned. Metric
FDs~\cite{koudas2009metric} allow small variations in the data,
% (similar
% attribute values are considered identical), 
but hinge on appropriate distance metrics and thresholds. For example, if
$\mathtt{time}$ is split across two attributes ($\mathtt{hour}$ and
$\mathtt{minute}$), the distance metric is non-trivial: it needs to encode that
$\langle \mathtt{hour} = 4, \mathtt{min} = 59 \rangle$ and $\langle
\mathtt{hour} = 5, \mathtt{min} = 1\rangle$ are similar, while $\langle
\mathtt{hour} = 4, \mathtt{min} = 1\rangle$ and $\langle \mathtt{hour} = 5,
\mathtt{min} = 59\rangle$ are not. In contrast, \dis can model the composite
attribute ($60 \cdot \mathtt{hour} + \mathtt{minute}$) by automatically
discovering the coefficients $60$ and $1$.
% for such a composite attribute.

\looseness-1 Denial constraints (DC)~\cite{DBLP:journals/pvldb/ChuIP13,
DBLP:journals/pvldb/BleifussKN17, pena2019discovery,
DBLP:journals/corr/abs-2005-08540} encapsulate a number of different
data-profiling primitives such as FDs and their variants (e.g,~\cite{
DBLP:conf/icde/FanGLX09}). Exact DCs can adjust to noisy data by adding
predicates until the constraint becomes exact over the entire dataset, but this
can make the constraint extremely large and complex, which might even fail to
provide the desired generalization. For example, a finite DC---whose language
is limited to universally quantified first-order logic---cannot model the
constraint of Example~\ref{ex:tml}, which involves an arithmetic expression
(addition and multiplication with a constant). Expressing \dis requires a
richer language that includes linear arithmetic expressions. \revisetwo{Pattern
functional dependencies (PFD)~\cite{qahtan2020pattern} move towards addressing
this limitation of DCs, but they focus on text attributes: they are regex-based
and treat digits as characters. However, modeling arithmetic relationships of
numerical attributes requires interpreting digits as numbers.}

\looseness-1 To adjust for noise, FDs and DCs either relax the notion of
constraint violation or allow a user-defined fraction of tuples to violate the
(strict) constraint~\cite{pena2019discovery, huhtala1999tane,
kruse2018efficient, DBLP:conf/sigmod/IlyasMHBA04, koudas2009metric,
caruccio2016discovery, DBLP:journals/corr/abs-2005-08540}. Some
approaches~\cite{DBLP:conf/sigmod/IlyasMHBA04, DBLP:conf/sigmod/ZhangGR20,
DBLP:conf/sigmod/YanSZWC20} use statistical techniques to model other types of
data profiles such as correlations and conditional dependencies. However, they
require additional parameters such as noise and violation thresholds and
distance metrics. In contrast, \dis do not require any parameter from the user
and work on noisy datasets.

\revisetwo{Existing data profiling techniques are not designed to learn what ML
models exploit and are sensitive to noise in the numerical attributes.
Moreover, data constraint discovery algorithms typically search over an
exponential set of candidates, and hence, are not scalable: their complexity
grows exponentially with the number of attributes or quadratically with data
size. In contrast, our technique for deriving \dis is highly scalable (linear
in data size) and efficient (cubic in the number of attributes). It does not
explicitly explore the candidate space, as PCA---which lies at the core of our
technique---performs the search \emph{implicitly} by iteratively refining
weaker \invariants to stronger ones.} \label{nocandidate}

\subsubsection*{Learning techniques} While \emph{ordinary least square} finds
the lowest-variance projection, it minimizes observational error on only the
target attribute, and, thus, does not apply to our setting. \emph{Total least
square} offers a partial solution as it takes observational errors on all
predictor attributes into account; but, it finds only one projection---the
lowest variance one---that fits the data tuples best. But there may exist other
projections with slightly higher variances and we consider them all. As we show
empirically in Section~\ref{exp-invariants-for-drift}, constraints derived from
multiple projections, collectively, capture various aspects of the data, and
result in an effective data profile targeted towards certain tasks such as
data-drift quantification~\citeTechRep.

\medskip 

\paragraph{Contributions.} We make the following contributions:

\begin{itemize}

     \item We ground the motivation of our work with two case studies on trusted
     machine learning (TML) and data drift. (Section~\ref{sec:casestudies})
     
     \item We introduce and formalize \dis, a new data profiling primitive that
     specifies constraints over arithmetic relationships among numerical
     attributes of a dataset. We describe a \emph{conformance language} to
     express \dis, and a \emph{quantitative semantics} to quantify how much a
     tuple violates the \dis. In applications of constraint violations, some
     violations may be more or less critical than others. To capture that, we
     consider a notion of \invariant importance, and weigh violations against
     \invariants accordingly. (Section~\ref{sec:data-invs})
      
     \item We formally establish that strong \dis are constructed from
     projections with small variance and small mutual correlation on the given
     dataset. Beyond simple linear \invariants (e.g., the one in
     Example~\ref{ex:tml}), we derive \emph{disjunctive} \invariants, which are
     disjunctions of linear \invariants. We achieve this by dividing the
     dataset into disjoint partitions, and learning linear \invariants for each
     partition. We provide an efficient, scalable, and highly parallelizable
     algorithm for computing a set of linear \dis and disjunctions over them.
     We also analyze its runtime and memory complexity.
     (Section~\ref{sec:synth-data-inv})
     
     \item We formalize the notion of \emph{\nc} tuples in the context of
     trusted machine learning and provide a mechanism to detect \nc tuples
     using \dis. (Section~\ref{sec:di-for-tml})
	 
     \item We empirically analyze the effectiveness of \dis in two case-study
     applications---TML and data-drift quantification. We show that \dis can
     reliably predict the trustworthiness of linear models and quantify data
     drift precisely, outperforming the state of the art.
     (Section~\ref{sec:experiments})
	 
  \end{itemize}