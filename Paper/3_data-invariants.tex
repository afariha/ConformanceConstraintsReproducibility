%!TEX root = paper.tex
%% \section{\DIs}

\def\Inv{\mathtt{Inv}}
\def\Dom{\mathtt{Dom}}
\def\coDom{\mathtt{coDom}}
\def\DDom{\mathbf{Dom}}
\def\CC{\mathcal{C}}
\def\Real{{\mathbb{R}}}
%\def\dist{{\mathtt{qmem}}}

\def\Bool{{\mathtt{Bool}}}
\def\true{{\mathtt{True}}}
\def\false{{\mathtt{False}}}
\def\vecw{{\vec{w}}}
\def\vecv{{\vec{v}}}
\def\lb{{\mathtt{lb}}}
\def\ub{{\mathtt{ub}}}
\def\atomic{{atomic}}
\def\Atomic{{Atomic}}
%\def\M{{\infty}}
%\def\Conj{{\varphi}}
%\def\Disj{{\Phi}}
\newcommand{\semq}[1]{{[\![{#1}]\!]}}
%\newcommand\avg[1]{{\overline{#1}}}
\newcommand\avg[1]{{\mathtt{\mu}}(#1)}
\newcommand\sign[1]{{\mathtt{sign}({#1})}}
\newcommand\delF[2]{{\Delta{#1}({#2})}}
\newcommand\stddev[1]{{\mathtt{\sigma}}(#1)}
\def\monthy{M}


In this section, 
%we define \dis that allow us to capture complex arithmetic dependencies involving numerical attributes of a dataset. 
we first provide the general definition of \dis. Then we propose a language for
representing them. Finally, we define quantitative semantics over \dis, which
allows us to quantify their violation.

% We present one possible \emph{language} for representing \dis,
% and attach a \emph{quantitative semantics} to it.
% The quantitative semantics, intuitively, computes the degree of
% \invariant violation on data.
%Finally, we provide a principle for synthesizing effective \dis. 
% Finally, we discuss how we prioritize
% \invariants during quantitative semantics computation based on their
%effectiveness.

\smallskip
% \noindent \textbf{tuple, attribute, and Dataset.}

\looseness-1
\paragraph{Basic notation.}
%
We use $\mathcal{R} (A_1, A_2, \dots, A_m)$ to denote a relation schema where
$A_i$ denotes the $i^{th}$ attribute of $\mathcal{R}$. We use $\Dom_i$ to
denote the domain of attribute $A_i$. Then the set ${\DDom}^m=\Dom_1\times
\cdots\times \Dom_m$ specifies the domain of all possible tuples. We use $t
\in\DDom^m$ to denote a tuple in the schema $\mathcal{R}$. A dataset
$D\subseteq {\DDom}^m$ is a specific instance of the schema $\mathcal{R}$. For
ease of notation, we assume some order of tuples in $D$ and we use $t_i \in D$
to refer to the $i^{th}$ tuple and $t_i.A_j \in \Dom_j$ to denote the value of
the $j^{th}$ attribute of $t_i$.

\ignore{
\paragraph{Tuple, Dataset, and Attribute.} Let $\Dom_1, \Dom_2, \ldots, \Dom_n$
denote some $n$ domains for values. We fix the set $\Dom_1\times \cdots\times
\Dom_n$ (written as ${\DDom}^m$) as the domain of all {\em{tuples}}. Each
\emph{tuple} $d\in\DDom^m$ is an $n$-tuple, where each component of the tuple
is an {\em{attribute}}.\footnote{\scriptsize{In this work, we use the term
``attribute'' to denote a single attribute.}} We use $A_1,\ldots,x_n$ to denote
the names of these $n$ attributes. We use $d_{x_j}\in \Dom_j$ to denote the
value of the $j^{th}$ attribute of $d$. A {\em{dataset}} $D\subseteq {\DDom}^m$
is a sequence of tuples, where the $i^{th}$ tuple is denoted using $d^{(i)}$.
\endignore
}

%We differentiate between two types of attributes in a dataset, namely
%categorical and numerical. An attribute with domain $\Dom_i$ is categorical if
%$\Dom_i$ is a finite discrete set with small (e.g., up to 50) cardinality. The
%domain of a numerical attribute is the set of numbers. 

% \sloppy
% \begin{definition}[\Di]\label{def:di}
%      %
%      A \di for a dataset $D \,{\subseteq}\, {\DDom}^m$ is another
%      dataset $\Inv \,{\subseteq}\, {\DDom}^m$ s.t. $D - {D'} \,{\subseteq}\,
%      \Inv$, and $\frac{|D'|}{|D|} \le \delta$.
%      %
% \end{definition}

\ignore{
\smallskip
\looseness-1 We start with a \emph{strict} definition of \dis, and then explain how it generalizes to account for noise in the data.
% 
\begin{definition}[\Di (strict)]\label{def:di}
     %
     A \di for a dataset $D \,{\subseteq}\, {\DDom}^m$ is a formula $\Phi:
     \DDom^m \mapsto \{\true,\false\}$ such that $\forall t\in D,
     \Phi(t) = \true$.
     %
\end{definition}
% 

\looseness-1
A \di ($\Phi$) characterizes a set of allowable or conforming tuples and is
expressed through a \emph{conformance language} (Section~\ref{sec:cl}). We write
$\Phi(t)$ and $\neg\Phi(t)$ to denote that $t$ satisfies and violates the
\invariant $\Phi$, respectively. In practice, because of noise, some tuples in
$D$ may violate a \invariant. To account for noise, we relax the definition of
\di as follows.
\endignore}
% %
% \looseness-1 We start with a formal definition of \dis, which is a formula
% expressed in a \emph{conformance language} (Section~\ref{sec:cl}),
% that is satisfied by almost all tuples in $D$.

\smallskip
% \noindent \textbf{tuple, attribute, and Dataset.}

\looseness-1 \paragraph{\Di.} A \di $\Phi$ characterizes a set of allowable or
conforming tuples and is expressed through a \emph{conformance language}
(Section~\ref{sec:cl}). We write $\Phi(t)$ and $\neg\Phi(t)$ to denote that $t$
satisfies and violates $\Phi$, respectively.

\begin{definition}[\Di]\label{def:di2}
     %
     A \di for a dataset $D \,{\subseteq}\, {\DDom}^m$ is a formula
     $\Phi: {\DDom}^m \mapsto \{\true,\false\}$ such that $|\{t \in D \mid
     \neg\Phi(t)\}| \ll |D|$.
	 %
\end{definition}
% 
The set $\{t \in D \mid \neg\Phi(t)\}$ denotes atypical tuples in $D$ that do
not satisfy the \di $\Phi$. In our work, we do not need to know the set of
atypical tuples, nor do we need to purge the atypical tuples from the dataset.
Our techniques derive \invariants in ways that ensure there are very few
atypical tuples (Section~\ref{sec:synth-data-inv}). 


\subsection{Conformance Language}\label{sec:cl}
%
\looseness-1 \paragraph{\View.} A central concept in our conformance language
is \linebreak \emph{\view}. Intuitively, a \view is a derived attribute that
specifies a ``lens'' through which we look at the tuples. More formally, a
\view is a function $F: \DDom^m \mapsto\Real$ that maps a tuple $t\in\DDom^m$
to a real number $F(t) \in \Real$. In our language for \dis, we only consider
\views that correspond to linear combinations of the numerical attributes of a
dataset. Specifically, to define a \view, we need a set of numerical
coefficients for all attributes of the dataset and the \view is defined as a
sum over the attributes, weighted by their corresponding coefficients. We extend
a projection $F$ to a dataset $D$ by defining $F(D)$ to be the sequence of
reals obtained by applying $F$ on each tuple in $D$ individually.

\smallskip

{
%\setlength{\abovedisplayskip}{0pt}
%\setlength{\belowdisplayskip}{0pt}
\paragraph{Grammar.} Our language for \dis consists of formulas
$\Phi$ generated by the following grammar:
%
{
\begin{eqnarray*}
    \phi & := & \lb \leq F(\vec{A}) \leq \ub \; \mid\; \wedge(\phi,\ldots,\phi)
    \\[-0.3em]
    \psi_A & := & \vee(\;(A = c_1) \maxand \phi,\; (A = c_2) \maxand \phi, \;\ldots)
    \\[-0.3em]
    \Psi & := & \psi_A \; \mid \; \wedge(\psi_{A_1}, \psi_{A_2}, \ldots)
	\\[-0.3em]
    \Phi & := & \phi  \;\mid\; \Psi
\end{eqnarray*}
}
%
% The language consists of
% (1)~equality constraints $A_j = c_j$ where $A_j$ is an attribute and $c_j \in \Dom_j$ is a concrete value;
% (2)~bounded \view constraints $\lb \leq F(\vec{A}) \leq \ub$ where
%     $F$ is a \view on $\DDom^m$, $\vec{A}$ is the tuple of formal
%     parameters $(A_1, A_2, \ldots, A_m)$, and $\lb, \ub \in \Real$ are reals; and
% (3)~their conjunctions ($\wedge$) and disjunctions ($\vee$).
%   %
  %
\looseness-1 
\noindent 
The language consists of (1)~bounded constraints $\lb \leq F(\vec{A})\leq \ub$
where $F$ is a \view on $\DDom^m$, $\vec{A}$ is the tuple of formal parameters
$(A_1, A_2, \ldots, A_m)$, and $\lb, \ub \in \Real$ are reals; (2)~equality
constraints $A = c$ where $A$ is an attribute and $c$ is a constant in
$A$'s domain; and (3)~operators ($\maxand$, $\wedge$, and $\vee$,) that connect
the constraints.
}
%where the first two are intuitively conjunctive and  and $\wedge$ are two conjunction operators and $\vee$ is a disjunctions ($\vee$).
  % conjunction $\wedge$ and disjunction $\vee$ operators are variable arity operators.
Intuitively, $\maxand$ is a switch operator that specifies which \invariant $\phi$
applies based on the value of the attribute $A$, $\wedge$ denotes conjunction,
and $\vee$ denotes disjunction.
% $\vee$ and $\wedge$ are the usual Boolean operators. We allow an
% equality $x = c$ to be part of a \di only when $x$ is categorical.
Formulas generated by $\phi$ and $\Psi$ are called {\em{simple \invariants}}
and {\em{compound \invariants}}, respectively. Note that a formula generated by
$\psi_A$ only allows equality constraints on a single attribute, namely $A$,
among all the disjuncts.
% Then $\Psi$ generates a
% \emph{compound \invariant} which is a conjunction over a set of disjunctive
% \invariants (on different attributes). Finally, $\Phi$ gives either a
% conjunctive \invariant or a compound \invariant.

\begin{example}\label{ex:constraints}
    Consider the dataset $D$ consisting of the first four tuples  $\{t_1, t_2, t_3, t_4\}$ of Fig.~\ref{fig:flights}.
    A simple \invariant for $D$ is:
	{
	$$
    \phi_1: -5\leq \mathtt{AT} - \mathtt{DT} - \mathtt{DUR} \leq 5.
	$$
	}
	%
    Here, the projection ${F(\vec{A})} = \mathtt{AT} - \mathtt{DT} - \mathtt{DUR}$, with attribute coefficients
    $\langle 1, - 1, -1 \rangle$, $\lb = {-}5$, and $\ub = 5$. A compound
    \invariant is:
	%
	{
	\begin{align*}
     \psi_2:   \mathtt{\monthy} &= \text{``May''}  \maxand -2 \leq \mathcolorbox{lightgray}{\mathtt{AT} - \mathtt{DT} - \mathtt{DUR}} \leq 0 \\[-0.4em]
     \vee \;\; \mathtt{\monthy} &= \text{``June''} \maxand \phantom{-}0  \leq \mathcolorbox{lightgray}{\mathtt{AT} - \mathtt{DT} - \mathtt{DUR}} \leq 5\\[-0.4em]
     \vee \;\; \mathtt{\monthy} &= \text{``July''} \, \maxand {-}5 \leq \mathcolorbox{lightgray}{\mathtt{AT} - \mathtt{DT} - \mathtt{DUR}} \leq 0
	\end{align*}
	}
	 %
     For ease of exposition, we assume that all times are converted to minutes
     (e.g., \texttt{06:10} $= 6 \times 60 + 10 = 370$) and $\monthy$ denotes
     the departure month, extracted from $Departure$ $Date$.
	 %
\end{example}

Note that arithmetic expressions that specify linear combination of numerical
attributes (highlighted above) are disallowed in denial
constraints~\cite{DBLP:journals/pvldb/ChuIP13} which only allow raw attributes
and constants (more details are in \appOrTechRep).




\subsection{Quantitative Semantics}\label{quant-sem}

\looseness-1 \Dis have a natural Boolean semantics: a tuple either satisfies a
\invariant or it does not. However, Boolean semantics is of limited use in
practice, because it does not quantify the degree of \invariant violation. We
interpret \dis using a quantitative semantics, which quantifies violations, and
% to handle noisy data.
% This is important because data can be noisy and
%Quantitative semantics has the additional benefit that it 
reacts to noise more gracefully than Boolean semantics.

%Given a formula $\Phi$,
% and a tuple $d$,
The quantitative semantics $\semq{\Phi}(t)$ is a measure of the violation of
$\Phi$ on a tuple $t$---with a value of $0$ indicating no violation and a value
greater than 0 indicating some violation. In Boolean semantics, if $\Phi(t)$ is
$\true$, then $\semq{\Phi}(t)$ will be $0$; and if $\Phi(t)$ is $\false$, then
$\semq{\Phi}(t)$ will be $1$. Formally, $\semq{\Phi}$ is a mapping from
$\DDom^m$ to $[0,1]$.

\smallskip

\noindent\emph{Quantitative semantics of simple \invariants.} 
%
\revisethree{We build upon $\epsilon$-insen\-sitive loss~\cite{vapnik1997support}
to define the quantitative semantics of simple \invariants, where the bounds
$\lb$ and $\ub$ define the $\epsilon$-insensitive
zone:}\footnote{\revisethree{For a target value $y$, predicted value $\hat{y}$,
and a parameter $\epsilon$, the $\epsilon$-insensitive loss is $0$ if $|y -
\hat{y}| < \epsilon$ and $|y - \hat{y}| - \epsilon$ otherwise.
% $\epsilon$-insensitive loss is defined by $
% %
%         \begin{cases}
%         0, & \text{if } |y - \hat{y}| < \epsilon\\
%         |y - \hat{y}| - \epsilon, & \text{otherwise}
%         \end{cases}
%   $
}}\label{fn1}
%
\begin{align*}
    \semq{\lb\leq F(\vec{A})\leq \ub}(t) &:= \eta(\alpha \cdot \max(0, F(t)-\ub, \lb - F(t)))
    \\
    \semq{\wedge(\phi_1,\ldots,\phi_K)}(t) &:= \textstyle\sum_{k}^{K} \gamma_k \cdot \semq{\phi_k}(t)
\end{align*}
%
Below, we describe the parameters of the quantitative semantics, and provide further details on them in \appOrTechRep.
% The quantitative semantics uses the following parameters:\footnote{More details on system parameters are in \appOrTechRep.}
% \begin{itemize}
%   \item a \emph{scaling factor} $\alpha\in\Real^+$,
%   \item a \emph{normalization function} $\eta(.): \Real\mapsto [0,1]$, and
%   \item \emph{importance factors} $\gamma_k \in \Real^+$ s.t. $\sum \gamma_k = 1$.
% \end{itemize}
% %and $\gamma_i$'s---which are non-negative real numbers whose sum is $1$---denote \emph{importance factors}.
% We discuss these parameters next.

\smallskip 
% \sloppy \noindent \textbf{Scaling factor $\alpha$.}
\paragraph{Scaling factor $\alpha\in\Real^+$.}\\ \Views are unconstrained
functions and different \views can map the same tuple to vastly different
values. We use a scaling factor $\alpha$ to standardize the values computed by
a \view $F$, and to bring the values of different \views to the same comparable
scale. The scaling factor is automatically computed as the inverse of the
standard deviation, $\alpha = \tfrac{1}{\stddev{F(D)}}$. 
We set $\alpha$ to a large positive number when $\stddev{F(D)}$ = $0$.

% A typical choice---inspired from the classical z-score normalization---is the inverse of standard deviation
% of $F(D)$, i.e., $1/\stddev{F(D)}$.\footnote{\scriptsize{When $\stddev{F(D)}$ = $0$, we use a large positive number $\M$ as the scaling factor.}}

\smallskip
% \sloppy \noindent \textbf{Normalization function $\eta$.}
\paragraph{Normalization function $\eta(.): \Real\mapsto [0,1]$.}\\
The normalization function
maps values in the range $[0,\infty)$ to the range $[0,1)$.
While any monotone mapping from $\Real^{\geq 0}$ to $[0,1)$ can be used, we
pick $\eta(z) = 1 - e^{-z}$.

\smallskip 
% \noindent\textbf{Importance factors $\gamma_i$.} 
\looseness-1
\paragraph{Importance factors $\gamma_k\in \Real^+$, $\textstyle\sum_{k}^{K}\gamma_k {=} 1$.}\\
The weights $\gamma_k$ control the contribution of each bounded-\view \invariant in a
conjunctive formula. This allows for prioritizing \invariants that are more
significant than others.
% within the context of a particular application.
In our work, we derive the importance factor of a \invariant automatically,
based on its \view's standard deviation over $D$.

\subsubsection*{Quantitative semantics of compound \invariants} 
Compound \invariants are first simplified into simple \invariants, and they get
their meaning from the simplified form.
We define a function
$\norm(\psi, t)$ that takes a compound \invariant $\psi$ and a tuple $t$ and returns a simple
\invariant. It is defined recursively as follows:
\begin{align*}
    &\norm(\vee(\;(A = c_1) \maxand\phi_1, (A = c_2) \maxand \phi_2,\ldots), t) := 
      \phi_k  \; \mbox{if $t.A = c_k$}
    \\
    &\norm(\wedge(\psi_{A_1}, \psi_{A_2},\ldots), t)  := \wedge(\norm(\psi_{A_1}, t), \norm(\psi_{A_2},t),\ldots)
\end{align*}
%
\begin{sloppypar} If the condition in the definition above does not hold for any $c_k$, then
$\norm(\psi,t)$ is undefined and $\norm(\wedge(\dots,\psi,\dots),t)$ is
also undefined. If $\norm(\psi,t)$ is undefined, then $\semq{\psi}(t) :=
1$. When $\norm(\psi,t)$ is defined, the quantitative semantics of $\psi$ is
given by:
$$
\semq{\psi}(t) := \semq{\norm(\psi,t)}(t)
$$
\end{sloppypar}

Since compound \invariants simplify to simple \invariants, we mostly focus on 
simple \invariants. Even there, we pay special attention to bounded-projection 
\invariants ($\phi$) of the form \linebreak $\lb \le F(\vec{A}) \le \ub$, which lie at the core of simple \invariants. 


\begin{example}\label{ex:semantics}
	%
    Consider the \invariant $\phi_1$ from Example~\ref{ex:constraints}.
	%
    % denoting $-5\leq F(\vec{A}) \leq 5$, where $F(\vec{A}) = AT - DT -
%     DUR$, over the database $D$ consisting of the first four tuples,
%     $\{t_1, t_2, t_3, t_4\}$, shown in Fig.~\ref{fig:flights}.
	%
	For $t \in D$,
    $\semq{\phi_1}(t)= 0$ since $\phi_1$ is satisfied by all tuples in $D$. The
    standard deviation of the projection $F$ over $D$, $\sigma(F(D)) {=}
    \sigma(\{0,-5, 5, -2\}) {=} 3.6$. Now consider the last tuple $t_5 \not\in
    D$. $F(t_5) = (370 - 1350)-458 = -1438$, which is way below the lower bound $-5$ of $\phi_1$. 
	Now we compute how much $t_5$ violates $\phi_1$: 
	$\semq{\phi_1}(t_5) = \semq{-5 \leq F(\vec{A}) \leq 5}(t_5) 
		 			   = \eta(\alpha \cdot \max(0, -1438 - 5, -5 + 1438)) = 1 - e^{-\frac{1433}{3.6}} \approx 1$.
%		 			   &= 1 - e^{-\frac{1433}{\sigma(F(D))}} = 1 - e^{-\frac{1433}{3.6}} \approx 1
%	\end{align*}
	%
	Intuitively, this implies that $t_5$ strongly violates $\phi_1$.
	%
\end{example}


